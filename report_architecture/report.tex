\documentclass{article}
\usepackage[utf8]{inputenc}
\usepackage{mathptmx}
\usepackage{amsmath}
\usepackage{amssymb}
\usepackage[top=2cm, bottom=2cm, left=2.5cm, right=2.5cm]{geometry}
\usepackage{setspace}
\usepackage[svgnames]{xcolor}
\usepackage{listings}
\usepackage{graphicx}
\usepackage{caption}
\usepackage{fancyhdr}
\usepackage{adjustbox}
\usepackage{import}
\usepackage{tcolorbox}
\usepackage{newfloat}
\usepackage{subcaption}
\usepackage{hyperref}

\sloppy

\newcommand{\RNum}[1]{\uppercase\expandafter{\romannumeral #1\relax}}
\newcommand{\risc}{RISC-\RNum{5}}

\title{Rapport architecture microprocesseur}
\author{Yannis Kedadry \and Oscar Garnier \and Antoine Anastassiades }
\date{12 decembre 2022}

\begin{document}
\maketitle

\section{Choix de l'architecture}
Nous avons choisi de réaliser notre projet en nous basant sur l'architecture de \risc{} 32 bits. Nous utiliserons 32 registres.

\section{Ensemble d'instructions}
Nous comptons utiliser un ensemble de \emph{32} instructions afin de réaliser des opérations sur des entiers non signés. Nous encoderons nos instructions de plusieurs façon différentes en fonction du type d'instruction (registres ou saut).

\begin{center}
	\def\arraystretch{1.5}
	\begin{tabular}{c*{5}{p{0.15\textwidth}}}
		&31\hfill20&19\hfill15&14\hfill10&9\hfill5&4\hfill0\\
		\cline{2-6}
		R&\multicolumn{1}{|c|}{immédiat}&\multicolumn{1}{|c|}{rs2}&\multicolumn{1}{|c|}{rs1}&\multicolumn{1}{|c|}{rd}&\multicolumn{1}{|c|}{opcode}\\
		\cline{2-6}
		I&\multicolumn{2}{|c|}{immédiat}&\multicolumn{1}{|c|}{rs}&\multicolumn{1}{|c|}{rd}&\multicolumn{1}{|c|}{opcode}\\
		\cline{2-6}
		U&\multicolumn{3}{|c|}{immédiat}&\multicolumn{1}{|c|}{rd}&\multicolumn{1}{|c|}{opcode}\\
		\cline{2-6}
\end{tabular}
\end{center}

\end{document}
