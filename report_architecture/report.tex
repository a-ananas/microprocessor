\documentclass{article}
\usepackage[utf8]{inputenc}
\usepackage{mathptmx}
\usepackage{amsmath}
\usepackage{amssymb}
\usepackage[top=2cm, bottom=2cm, left=2.5cm, right=2.5cm]{geometry}
\usepackage{setspace}
\usepackage[svgnames]{xcolor}
\usepackage{listings}
\usepackage{graphicx}
\usepackage{caption}
\usepackage{fancyhdr}
\usepackage{adjustbox}
\usepackage{import}
\usepackage{tcolorbox}
\usepackage{newfloat}
\usepackage{subcaption}
\usepackage{hyperref}

\sloppy

\newcommand{\RNum}[1]{\uppercase\expandafter{\romannumeral #1\relax}}
\newcommand{\risc}{RISC-\RNum{5}}

\title{Report microprocessor architecture}
\author{Yannis Kedadry \and Oscar Garnier \and Antoine Anastassiades }
\date{\today}

\begin{document}
\maketitle

\section{Architecture choice}
We've chosen to make our project using the 32 bits \risc{}'s architecture. We'll use 32 registers.

\section{Instruction Set}
We'll use a total of 28 instructions to make operations between unsigned integers. We'll encode our instructions in different
ways according to the instruction's format (operation with immediates I, with jumps U, only on registers R).
The different operations we'll use are:
\begin{itemize}
	\item \emph{add}, \emph{sub} (type R) and \emph{addi} (type I). Our arithmetical instructions.
	\item \emph{sll}, \emph{srl}, \emph{sra}  Of type R and their equivalent of type I to make bitwise shifts.
	\item logical operations \emph{and}, \emph{or} and \emph{xor} of type R and their equivalent of type I.
	\item tests operations \emph{slt} and \emph{slti}.
	\item \emph{lui} and \emph{auipc} are special instructions of type U, used to move in the stack.
	\item \emph{beq}, \emph{bne}, \emph{blt}, \emph{blti} (type I) and \emph{bge} for conditional branching. Except for \emph{blti}, they're of type R.
	\item \emph{jal} and \emph{jalr} of type U for unconditional jumps.
	\item \emph{lw} (type I) and \emph{sw} (type R) load and store datas from and to memory.
\end{itemize}

Here are the different encoding for the different instruction formats. These formats help us manipulate immediates of big size.
It's probably not mandatory but it can come handy in some particular cases.

\emph{rd} represents the destination register and \emph{rsX} the sources ones.
\begin{center}
	\def\arraystretch{1.5}
	\begin{tabular}{c*{5}{p{0.15\textwidth}}}
		&31\hfill20&19\hfill15&14\hfill10&9\hfill5&4\hfill0\\
		\cline{2-6}
		R&\multicolumn{1}{|c|}{immédiat}&\multicolumn{1}{|c|}{rs2}&\multicolumn{1}{|c|}{rs1}&\multicolumn{1}{|c|}{rd}&\multicolumn{1}{|c|}{opcode}\\
		\cline{2-6}
		I&\multicolumn{2}{|c|}{immédiat}&\multicolumn{1}{|c|}{rs}&\multicolumn{1}{|c|}{rd}&\multicolumn{1}{|c|}{opcode}\\
		\cline{2-6}
		U&\multicolumn{3}{|c|}{immédiat}&\multicolumn{1}{|c|}{rd}&\multicolumn{1}{|c|}{opcode}\\
		\cline{2-6}
	\end{tabular}
\end{center}

\newpage
\section{Opcodes}

Our instructions will have the following opcodes:
\newline

\begin{center}
	\def\arraystretch{2.5}
	\begin{tabular}{p{0.08\textwidth}||c*{8}{p{0.08\textwidth}}}
		& 000 & 001 & 010 & 011 & 100 & 101 & 110 & 111 \\
		\cline{1-9}
		00 & Addi & & Srli & Srai & Slli & Andi & Ori & Xori \\
		\cline{1-9}
		01 & Add & Sub & Srl & Sra & Sll & And & Or & Xor \\
		\cline{1-9}
		10 & Lw & Sw & & Beq & Bne & Blt & Blti & Bge \\
		\cline{1-9}
		11 & Lui & Auipc & Jal & Jalr & & & Slt & Slti \\
	\end{tabular}
\end{center}

\section{TLDR}

% array
% instruction | format | opcode | usage
% addi | I | 00000 | addi x0 x1 1 => xo = x1 + 1
\begin{center}
	\def\arraystretch{2.2}
	\begin{tabular}{c|c|c|c|c}
		instruction & format & opcode & usage & result\\
		\cline{1-5}
		Addi & I & 00000 & addi x0 x1 1 & x0 = x1 $+$ 1 \\
		Srli & I & 00010 & srli x0 x1 1 & x0 = x1 $>>$ 1 (0 extended)\\
		Srai & I & 00011 & srai x0 x1 1 & x0 = x1 $>>$ 1 (sign extended)\\
		Slli & I & 00100 & slli x0 x1 1 & x0 = x1 $<<$ 1 \\
		Andi & I & 00101 & andi x0 x1 1 & x0 = x1 \& 1 \\
		Ori & I & 00110 & ori x0 x1 1 & x0 = x1 $|$ 1 \\
		Xori & I & 00111 & xori x0 x1 1 & x0 = x1 $\oplus$ 1 \\
		Add & R & 01000 & add x0 x1 x2 1& x0 = x1 $+$ x2 \\
		Sub & R & 01001 & sub x0 x1 x2 1& x0 = x1 $-$ x2 \\
		Srl & R & 01010 & srl x0 x1 x2 1& x0 = x1 $>>$ x2 (0 extended)\\
		Sra & R & 01011 & sra x0 x1 x2 1& x0 = x1 $>>$ x2 (sign extended)\\
		Sll & R & 01100 & sll x0 x1 x2 1& x0 = x1 $<<$ x2 \\
		And & R & 01101 & and x0 x1 x2 1& x0 = x1 \& x2 \\
		Or & R & 01110 & or x0 x1 x2 1& x0 = x1 $|$ x2 \\
		Xor & R & 01111 & xor x0 x1 x2 1& x0 = x1 $\oplus$ x2 \\
		Lw & I & 10000 & lw x0 x1 1 & x0 = Ram(x1)\\
		Sw & R & 10001 & sw x0 x1 x2 1& Ram(x2) = x1\\
		Beq & R & 10011 & beq x0 x1 x2 1& rdi = (x1 $=$ x2) ? rdi + 1 : rdi + 4\\
		Bne & R & 10100 & bne x0 x1 x2 1& rdi = (x1 $\neq$ x2) ? rdi + 1 : rdi + 4\\
		Blt & R & 10101 & blt x0 x1 x2 1& rdi = (x1 $<$ x2) ? rdi + 1 : rdi + 4\\
		% Blti & I & 10110 & blti x0 x1 1& rdi = (x1 $<$ 1) ? rdi + 1 : rdi + 4\\
		Bge & R & 10111 & bge x0 x1 x2 1& rdi = (x1 $\geq$ x2) ? rdi + 1 : rdi + 4\\
		% Lui
		% Auipc
		Jal & U & 11010 & jal x0 1 & rdi = 1\\
		% Jalr & U & 11011 & jalr x0 1  & rdi = 1\\
		Slt & R & 11110 & slt x0 x1 x2 3 & x0 = (x1 $<$ x2) ? 1 : 0 \\
		Slti & I & 11111 & slti x0 x1 3 & x0 = (x1 $<$ 3) ? 1 : 0 \\
	\end{tabular}
\end{center} 

\end{document}
