\documentclass{article}
\usepackage[utf8]{inputenc}
\usepackage{mathptmx}
\usepackage{amsmath}
\usepackage{amssymb}
\usepackage[top=2cm, bottom=2cm, left=2.5cm, right=2.5cm]{geometry}
\usepackage{setspace}
\usepackage[svgnames]{xcolor}
\usepackage{listings}
\usepackage{graphicx}
\usepackage{caption}
\usepackage{fancyhdr}
\usepackage{adjustbox}
\usepackage{import}
\usepackage{tcolorbox}
\usepackage{newfloat}
\usepackage{subcaption}
\usepackage{hyperref}

\sloppy

\newcommand{\RNum}[1]{\uppercase\expandafter{\romannumeral #1\relax}}
\newcommand{\risc}{RISC-\RNum{5}}

\title{Rapport architecture microprocesseur}
\author{Yannis Kedadry \and Oscar Garnier \and Antoine Anastassiades }
\date{12 decembre 2022}

\begin{document}
\maketitle

\section{Choix de l'architecture}
Nous avons choisi de réaliser notre projet en nous basant sur l'architecture de \risc{} 32 bits. Nous utiliserons 32 registres.

\section{Ensemble d'instructions}
Nous comptons utiliser un ensemble de \emph{32} instructions afin de réaliser des opérations sur des entiers non signés. Nous encoderons nos instructions de plusieurs façon différentes en fonction du type d'instruction (registres ou saut).

\end{document}
