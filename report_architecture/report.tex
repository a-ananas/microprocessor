\documentclass{article}
\usepackage[utf8]{inputenc}
\usepackage{mathptmx}
\usepackage{amsmath}
\usepackage{amssymb}
\usepackage[top=2cm, bottom=2cm, left=2.5cm, right=2.5cm]{geometry}
\usepackage{setspace}
\usepackage[svgnames]{xcolor}
\usepackage{listings}
\usepackage{graphicx}
\usepackage{caption}
\usepackage{fancyhdr}
\usepackage{adjustbox}
\usepackage{import}
\usepackage{tcolorbox}
\usepackage{newfloat}
\usepackage{subcaption}
\usepackage{hyperref}

\sloppy

\newcommand{\RNum}[1]{\uppercase\expandafter{\romannumeral #1\relax}}
\newcommand{\risc}{RISC-\RNum{5}}

\title{Rapport architecture microprocesseur}
\author{Yannis Kedadry \and Oscar Garnier \and Antoine Anastassiades }
\date{12 decembre 2022}

\begin{document}
\maketitle

\section{Choix de l'architecture}
Nous avons choisi de réaliser notre projet en nous basant sur l'architecture de \risc{} 32 bits. Nous utiliserons 32 registres.

\section{Ensemble d'instructions}
Nous comptons utiliser un ensemble de 28 instructions afin de réaliser des opérations sur des entiers non signés. 
Nous encoderons nos instructions de plusieurs façon différentes en fonction du type d'instruction 
(opérations avec immédiats I, avec saut U ou uniquement sur des registres R). 
Les différentes opérations que nous utiliserons seront : 
\begin{itemize}
	\item \emph{add}, \emph{sub} (types R) et \emph{addi} (type I). Ce sont nos instructions arithmétiques.
	\item \emph{sll}, \emph{srl}, \emph{sra} de types R et leur pendant de type I afin de réaliser des décalages bit à bit.
	\item les opérations logiques \emph{and}, \emph{or} et \emph{xor} de type R et leur pendant de type I.
	\item les opérations de tests \emph{slt} et \emph{slti}.
	\item \emph{lui} et \emph{auipc} sont des instructions spéciales permettant de se déplacer sur la pile. Ce sont des instructions de type J.
	\item \emph{beq}, \emph{bne}, \emph{blt}, \emph{blti} et \emph{bge} nous permettent de réaliser des branchements conditionnels. Ils sont de type I.
	\item \emph{jal} et \emph{jalr} permettent des branchements inconditionnels. Ils sont de type J.
	\item \emph{lw} charge \emph{sw} stocke des données en mémoire. Ils sont de type I.
\end{itemize}

Voici les différents encodage pour les différents types d'instructions. Avoir ces types nous permet de manipuler des immédiats de grande taille. Ce n'est peut-être pas impératif mais cela pourrait nous faciliter la tâche dans certains cas précis.
\emph{rd} représente les registres de destination et \emph{rsX} les registre sources qui peuvent être unique ou en pair.
\begin{center}
	\def\arraystretch{1.5}
	\begin{tabular}{c*{5}{p{0.15\textwidth}}}
		&31\hfill20&19\hfill15&14\hfill10&9\hfill5&4\hfill0\\
		\cline{2-6}
		R&\multicolumn{1}{|c|}{immédiat}&\multicolumn{1}{|c|}{rs2}&\multicolumn{1}{|c|}{rs1}&\multicolumn{1}{|c|}{rd}&\multicolumn{1}{|c|}{opcode}\\
		\cline{2-6}
		I&\multicolumn{2}{|c|}{immédiat}&\multicolumn{1}{|c|}{rs}&\multicolumn{1}{|c|}{rd}&\multicolumn{1}{|c|}{opcode}\\
		\cline{2-6}
		U&\multicolumn{3}{|c|}{immédiat}&\multicolumn{1}{|c|}{rd}&\multicolumn{1}{|c|}{opcode}\\
		\cline{2-6}
	\end{tabular}
\end{center}

\newpage
\section{Opcodes}

Nos instructions suiveront les opcodes suivant:
\newline

\begin{center}
	\def\arraystretch{2.5}
	\begin{tabular}{p{0.08\textwidth}||c*{8}{p{0.08\textwidth}}}
		& 000 & 001 & 010 & 011 & 100 & 101 & 110 & 111 \\
		\cline{1-9}
		00 & Addi & & Srli & Srai & Slli & Andi & Ori & Xori \\
		\cline{1-9}
		01 & Add & Sub & Srl & Sra & Sll & And & Or & Xor \\
		\cline{1-9}
		10 & Lw & Sw & & Beq & Bne & Blt & Blti & Bgr \\
		\cline{1-9}
		11 & Lui & Auipc & & Jal & Jalr & & Slt & Slti \\
	\end{tabular}
\end{center}



\end{document}
